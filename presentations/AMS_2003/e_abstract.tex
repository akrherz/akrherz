\documentclass[twocolumn]{article}

\usepackage[conf]{ametsoc}


\begin{document}
\title{\bf 6.4      USING GIS FOR ENVIRONMENTAL DATA IN IOWA}
\author{D. P. Todey and D. Herzmann \thanks{Corresponding Author:  Daryl E. Herzmann, 3010
Agronomy Hall, Iowa State University, Ames, IA Email: akrherz@iastate.edu}\\
 Department of Agronomy, Iowa State University, Ames, Iowa \\
}

\amstitle

\section{Introduction}

The Iowa Environmental Mesonet (IEM hereafter) is a project at Iowa State 
University aiming to gather, compare, disseminate and archive environmental 
data from the state of Iowa.  Geographic Information Systems (GIS) have
provided the IEM an invaluable mechanism for data visualization 
and distribution.  This abstract highlights some of ways the IEM is using
GIS to distribute environmental data in Iowa.

\section{Technical Components}

Data arriving at the IEM servers come in many formats.  None of these formats
 would be classified as a true GIS datatype.  Via a handful of conversion
scripts, the various data sources are eventually stored in a relational 
database.  We choose PostgreSQL because of its open-source 
license and enterprise functionality.  Within the database, PostgreSQL has 
an extension called PostGIS, which spatially enables PostgreSQL.  With this 
extension, the PostgreSQL database can act as a backend spatial datasource for
GIS.

With the IEM data in a spatial database, many GIS applications are able to 
dynamically query out data they need.  The IEM uses a web-mapping GIS 
application called MapServer to produce web graphics and display query results.
MapServer is an OpenSource environment for building spatially enabled web
applications.  Utilizing other OpenSource technologies, MapServer is a fast 
cross platform GIS.

The IEM also uses an extension to MapServer called MapScript.  In particular, 
we use an extension to PHP to provide MapServer functionality in PHP.  PHP
is a popular scripting languauge used primarily for web development.  The 
combination of PHP and MapServer creates a terrific environment to build GIS
applications.

\section{MapServer Applications}

To date, the IEM has produced over twenty web-based GIS applications using 
MapServer.  It is our intention to continue to build more applications and
refine the ones we have currently built.  For the sake of brevity, we will
highlight four of these applications.

\subsection{Iowa Tornado Database}

A climatology of tornado reports in Iowa was generated based on a dataset 
provided by the Storm Prediction Center in Norman, Oklahoma.  This dataset was 
massaged and placed into a spatial database for further manipulation and 
display. MapServer easily combined this datasource with other GIS layers to 
produce a zoomable and queriable web application.

Using the configurability of MapServer, tornados with different strengths 
are segregated using different colors and 'killer tornados' can be 
individually plotted as well.  MapServer also allows the user to query 
the generated image for individual meta-data about points 
(tornadic events) plotted in the image.

Tornado reports were the first step in our effort to convert other hazzards
into a GIS database.  Having the reports in a GIS container will open the 
door to many interesting comparisons.  For example, a spatial correlation
between tornado sitings and major roadways.  It will also make the dataset
much more available for other users in the state.

\subsection{Where's it raining?}

The IEM collects environmental data from local school networks operated by
KCCI-TV (Des Moines, Ia) and KELO-TV (Sioux Falls, SD).  These networks report
live, up to the minute precipitation numbers to the IEM server.  A process
monitors this datafeed and generates 15 minute accumulations in real-time.

With the help of GEMPAK, another process converts recent NEXRAD imagery into
GeoTIFFs.  GeoTIFFs are TIFF format images with a spatial reference attached
to them.  Due to the lack of a tool to automatically do this conversion, 
scripts and rough fitting are used to spatially align the image to a projection.

MapServer, via a web interface, combines the NEXRAD imagery together with the
15 minute accumulations to produce a qualitative comparison of the base
reflectivity product (\#19) and accumulations.  This application has been 
useful in cases of virga and general rainfall events.  It is also helpful when
it is not raining, since the accumulation will look suspect on the map.

\subsection{NWS COOP Climatology}

The NWS COOP database provides the official climatological record. 
Unfortunately, the formats distributed by NCDC are typically one-step away 
from being in GIS format.  For many users, this first step is too large.  The
IEM, in many cases, acts as a data bridge between those with data and those
needing data.  Providing the NWS COOP dataset in GIS format is example of 
our mission.

With the help of the spatial database, the COOP data is easily queriable by
MapServer and available for many other applications that connect to the
database.  Many of the resulting dataset plots provide an interesting picture
of climatology.  For instance, plotting the record low temperature for a day 
and the year of the record produces a plot showing the spatial distribution 
of temperatures and the year when it happened.


\subsection{Data Map Generation}

Visualization is a powerful tool for most types of story telling.  In 
Meteorology, the theorem holds that a picture is worth a thousand words.  For
example, NEXRAD base reflectivity is represented with images and not textual 
tables.  Many data users of the IEM are only interested in quick and clean 
data displays of a variable across the state, ie temperature.

With the help of an extension to MapServer called PHP MapScript, dynamic GIS 
plots are generated within seconds of the user's selection.
  Since the application is GIS based, it makes it very easy to incorperate 
other GIS layers such as road databases and river basins.  For example, we have
combined a NEXRAD reflectivity layer with a GIS layer containing NWS Weather Forecast Office (WFO) County Warning Area (CWA) shapes with current county and 
polygon warnings highlighted.  This plot can be looped over time to produce
an interesting look at cell propogation within a warning box.


\section{GIS-ready NEXRAD imagery}
% http://ams.confex.com/ams/annual2003/19IIPS/abstracts/57902.htm

The IEM was first exposed to GIS after receiving many requests for real-time
NEXRAD information in various GIS applications.  Users were most interested in
viewing reflectivity products and NEXRAD precipiation estimations. After 
some investigation, we found it relatively easy to geo-reference 
reflectivity images and statically provide the data to users.

Some users are not able to handle imagery, so they wanted the NEXRAD data in
ESRI shapefile format.  NEX2SHP, an open-source encoder written by Scott Shipley,
 is currently being used to convert real-time imagery into shapefile format.
Future versions of NEX2SHP will also include elevation information and the 
polygons of the NEXRAD shapes.  

The real challenge has been how to stream this information to users in real-time
and avoid the static downloads.  It also seems difficult to include timestamp
information with the products.  We are looking for ways to meet these 
challenges in the near future.

\section{Conclusions}

Using the spatial extension, PostGIS, to the PostgreSQL database, the IEM has
been able to distribute environmental data in the state of Iowa.  With the help
of MapServer, we have been able to generate dynamic GIS plots of datasets for
fast and clean viewing by our users.  Using PHP MapScript, an extension for 
MapServer, we have been able to generate very customizable plotting applications
and data visualization applications.

The IEM has just tipped the iceberg with regards to the use of GIS with
environmental data in the Iowa.  We are the first ones to admit that we 
are not using GIS to its full capability. As we continue to build 
partnerships and  collaborations, new ideas to utilize GIS will 
formulated and it be exciting to see where our efforts take us.

\end{document}
