\documentstyle[11pt]{article}
\renewcommand{\oddsidemargin}{0pt}
\renewcommand{\evensidemargin}{0pt}
\renewcommand{\topmargin}{0pt}
\renewcommand{\headheight}{0pt}
\renewcommand{\headsep}{0pt}
\renewcommand{\textwidth}{468pt}
\renewcommand{\textheight}{260mm}
\hoffset=-60pt
\voffset=-36pt
%
\settowidth{\parindent}{}
\setlength{\parskip}{2mm}
%
\newcommand{\dvd}{\rule{30mm}{0.2mm}}
\newcommand{\beq}{\begin{equation}}
\newcommand{\beqn}{\begin{eqnarray}}
\newcommand{\eeq}{\end{equation}}
\newcommand{\eeqn}{\end{eqnarray}}
\newcommand{\lsf}{\large \sf}
%
\newcommand{\qbar}{\mbox{$\overline{q}$}}
% \pagestyle{empty}
\pagenumbering{arabic}
%
\begin{document}
%
%\twocolumn
%____________________________________________________________________________
\vspace*{0.25in}
\begin{center}
\LARGE \bf Daryl E. Herzmann\\
\Large \bf Curriculum Vitae\\
\end{center}
\begin{tabbing}
\= \Large \bf Contact Information \hspace*{2.5in} \= \\
\> \> \\
\> \lsf Department of Agronomy \> \lsf Phone : (515)294-5978\\
\> \lsf Iowa State University \> \lsf Fax: (515)294-3163\\
\> \lsf Ames, IA 50011 \> \lsf Email: akrherz@iastate.edu \\
\end{tabbing}
\Large \bf Education
\normalsize \sf
\begin{itemize}
\item B.A., 2001, Iowa State University.  Meteorology Major.
\end{itemize}
%\vspace*{0.25in}
\Large \bf Employment\\ \\
\normalsize \bf Iowa State University, October 1997 - present
\normalsize \sf
\begin{itemize}
\item August 2001 - present: Assistant Scientist, Iowa State University.\\
 \textbf{Supervisors}: Dr Dennis Todey (2001-2004), Dr Raymond Arritt (2004-2018) and Dr Rick Cruse (2018-).  \\
\textbf{Iowa Environmental Mesonet}: Develop an environmental data collection
platform to support the research and outreach activities of the University.
Determine and execute new approaches to data collection and dissemination 
through a feature rich website.  Manage the day to day support demands 
created by the large number of core partners and general public. \\
\textbf{Department of Agronomy}: Collaborate on multi-disciplinary research
times providing computational and meteorological expertise to the groups. \\
\textbf{Grant-Supported Research} Manage High Performance Computing platforms
to support the research activities of Ag-Meteorology group.  Design, 
implement, and administer computing and storage platforms.
\item March 1998 - May 2001: International Institute for Theoretical \& Applied 
Physics (IITAP).  Supervisor: Doug Fils. Duties included multi-platform 
systems administration, computer programming and computer instruction.
\item October 1997 - May 2001: Partnerships to Advance Learning in Science
(PALS).  Supervisor: Dr Doug Yarger.  Duties included website administration,
database programming and data manipulation.
\end{itemize}
%end employment
%
\Large \bf Computational Skills
\normalsize \sf
\begin{itemize}
\item IT Cerifications: Red Hat Certified Engineer (RHCE for Red Hat 6).
\item Software Graphing packages: NCL, GrADS, Matplotlib.
\item Data Markup and Computer Scripting Languages: HTML5, XML, JSON, shell scripting.
\item Biophysical Software Models: AERMOD (EPA Regulatory Model), WEPP (Soil Erosion Model).
\item Meteorological Software Models: MM5 (Weather/Climate Simulator), WRF (Weather/Climate Simulator).
\item Programming Languages: Java, PHP, Python, SQL.
\end{itemize} 
% End of Skills
%
\normalsize \sf
\Large \bf Awards
\normalsize \sf
\begin{itemize}
\item Operational Achievement Award, National Weather Association. (2012)
\item Environmental Hero Award, National Oceanic and Atmospheric Administration. (2007)
\item Larry R Johnson Award, National Weather Association. (2002)
\end{itemize}



\vspace*{0.25in}
\Large \bf Current Research Projects (PIs, Team Size)
\normalsize \sf
\begin{itemize}
\item Iowa Environmental Mesonet (2, 2)\\
2001-2019. Integrating diverse datasets from eight
different observing networks in the state of Iowa.  Program and maintain data
ingestion / manipulation systems to process the data. Exploring and implementing 
new techniques for data visualization and dissemination.  Coordinate with local
agencies with oversight of their observing networks.  Data focalpoint for 
observations in the state of Iowa. Project website generates billions of hits 
from millions of unique internet addresses (users) per month.  Email and 
phone support is given for hundreds of agencies that contact the project each
year.
\item Daily Erosion Project (2, 8)\\
2014-2019. Developing a model system to analyze real-time soil erosion loss over the 
midwestern US.  Responsibilities include the development of the model execution
environment, generation of analysis products, and dissemination of products 
via website.
\item Sustainable Corn and Cropping Systems (38, 170)\\
2011-2017. [20 million dollar grant by USDA-NIFA] Construction and management of multi-disclipinary database supporting collection, analysis, and modelling efforts of project. 
Evaluate, determine, and implement virtual collaboration platforms to 
support project collaboration.  Provide meteorological and climate modeling
expertise to support synthesis.
\end{itemize}
\vspace*{0.25in}
\Large \bf Past Research Projects (PIs, Team Size)
\normalsize \sf
\begin{itemize}
\item Iowa Department of Transportation (1, 2)\\
Created a software system to generate near-realtime forecasts of pavement
temperatures and conditions to support winter activities of the DOT.  Created
two new interfaces between weather forecast data and domain-specific forecast
models.
\item Wind-Energy Research with Industry Partner (1, 6)\\
Curated a very large and confidential dataset from a wind farm in Iowa. Wrote
analysis routines and a web interface to allow collaborators to query and
visualize the dataset.
\item Mississippi Department of Transporation (3, 6)\\
Generated domain-specific pavement performance model weather input files
to facilitate evaluation of current and future climates impact on 
Mississippi Roads.
\item Multi-RCM Ensemble Downscaling (1, 3)\\
Assisted with the proctoring of climate models on High Performance Computing.
Supported the researchers involved in analysis and diagnosis of output.
\item North American Regional Climate Change Assessment Program (6, 100+)\\
Proctored the simulation of climate models on High Performance Computing.
 Analyzed the output and prepared model output summary files as per 
community standards.  Assisted in the development and debugging of model
codes for this effort.
\item Pacific Climate Change Science Program (3, 6)\\
Proctored the simulation of climate models on High Performance Computing.
Designed and constructed a computing cluster to support this effort.
\end{itemize}
%
\Large \bf ISU Internal Committee Memberships
\normalsize \sf
\begin{itemize}
\item High Performence Computing - Storage Subcommittee
\item Red Hat Enterprise Linux Management Team
\end{itemize}
%
\Large \bf Membership in Professional Societies
\normalsize \sf	
\begin{itemize}
\item American Meteorological Society - Severe Local Storms Committee (2013-2015)
\item National Weather Association
\end{itemize}
%
\Large \bf Websites Created and Currently Maintained
\normalsize \sf
\begin{itemize}
\item Daily Erosion Project (v2). https://dailyerosion.org
\item Iowa State Data Team. https://datateam.agron.iastate.edu
\item Iowa Environmental Mesonet. https://mesonet.agron.iastate.edu
\item Sustainable Corn. https://sustainablecorn.org
\end{itemize}
\Large \bf Administrative Support for Websites
\normalsize \sf
\begin{itemize}
\item ISU Atmospheric Sciences Homepage. http://www.meteor.iastate.edu
\item ITS ClassNet. http://classnet.geol.iastate.edu
\item Meteorology Archive Server. http://mtarchive.geol.iastate.edu
\item Meteorology Current Data Server. http://metfs1.agron.iastate.edu
\end{itemize}
%
\Large \bf Publications\\ \\
\normalsize \bf Papers
\normalsize \sf
\begin{itemize}
%\item Gallus Jr., W. A., D. N. Yarger, D. E. Herzmann, 2000: An Interactive 
%Severe Weather Activity to Motive Student Learning. Bulletin of the American 
%Meteorological Society: Vol. 81, No. 9, pp. 2205-2212.
\item Iqbal, Javed, M. Necpalova, S. Archontoulis, R. Anex, M. Bourguignon, \textbf{D. Herzmann}, D. Mitchell, J. Sawyer, Q. Zhu, and M. Castellano. Extreme weather-year sequences have nonadditive effects on environmental nitrogen losses. Global Change Biology, 2017; 24 e303-e317. http://dx.doi.org/10.1111/gcb.13866
\item Valcu-Lisman, A.M., P.W. Gassman, R. Arritt, T. Campbell, and \textbf{D.E. Herzmann}. Cost-effectiveness of reverse auctions for watershed nutrient reductions in the presence of climate variability: An empirical approach for the Boone River watershed. Journal of Soil and Water Conservation, 2017; 72(3) 280-295. http://dx.doi.org/10.2489/jswc.72.3.280
\item Hornbuckle B, J. Patton, A. VanLoocke, A. Suyker, M. Roby, V. Walker, E. Iyer, D. Herzmann, and E Endacott. SMOS optical thickness changes in response to the growth and development of crops, crop management, and weather. Remote Sensing of Environment, 2016; 180 320–333. http://dx.doi.org/10.1016/j.rse.2016.02.043
\item Panagopoulos Y, Gassman P W, Arritt R W, \textbf{Herzmann D E}, Campbell T D, Valcu A, et al.  Impacts of climate change on hydrology, water quality and crop productivity in the Ohio-Tennessee River Basin.  Int J Agric & Biol Eng, 2015; 8(3)
\item \textbf{Herzmann, D.}, L. Abendroth, and L. Bunderson. 2014. Data management approach to multidisciplinary agricultural research and syntheses. Journal of Soil and Water Conservation, 69(6), 180A-185A. http://dx.doi.org/10.2489/jswc.69.6.180A
\item Panagopoulos, Y., P. Gassman, R. Arritt, \textbf{D. Herzmann}, T. Campbell, M. Jha, C. Kling, R. Srinivasan, M. White, and J. Arnold. 2014. Surface water quality and cropping systems sustainability under a changing climate in the Upper Mississippi River Basin, Journal of Soil and Water Conservation, 69(6), 483-494. http://dx.doi.org/10.2489/jswc.69.6.483
\item M. Necpálová, R.P. Anex, A.N. Kravchenko, L.J. Abendroth, S.J. Del Grosso, W.A. Dick, M.J. Helmers, \textbf{D. Herzmann}, J.G. Lauer, E.D. Nafziger, J.E. Sawyer, P.C. Scharf, J.S.Strock and M.B. Villamil. 2014. What does it take to detect a change in soil carbon stock? A regional comparison of minimum detectable difference and experiment duration in the North-Central United States. Journal of Soil and Water Conservation, 69(6), 517-531. http://dx.doi.org/10.2489/jswc.69.6.517
\item Kladivko, E.J., M.J. Helmers, L.J. Abendroth, \textbf{D. Herzmann}, R. Lal, M. Castellano, D.S. Mueller, J.E. Sawyer, R.P. Anex, R.W. Arritt, B. Basso, J.V. Bonta, L. Bowling, R.M. Cruse, N.R. Fausey, J. Frankenberger, P. Gassman, A.J. Gassmann, C.L. Kling, A. Kravchenko, J.G. Lauer, F.E. Miguez, E.D. Nafziger, N. Nkongolo, M. O'Neal, L.B. Owens, P. Owens, P. Scharf, M.J. Shipitalo, J.S. Strock and M.B. Villamil. 2014. Standardized research protocols enable transdisciplinary research of climate variation impacts in corn production systems.Journal of Soil and Water Conservation, Special Issue for Climate and Agriculture. 69(6), 532-542. http://dx.doi.org/10.2489/jswc.69.6.532
\item Panagopoulos, Y., P.W. Gassman, R.W. Arritt, \textbf{D.E. Herzmann}, T.D. Campbell, A. Valcu, M.K. Jha, C.L. Kling, R. Srinivasan, M. White and J.G. Arnold. 2014. Hydrologic, water quality and crop productivity impacts of climate change in the Ohio-Tennessee river basin. International Journal of Agricultural and Biological Engineering. In-review.
\item Khanal, S., R. Anex, C. Anderson, and \textbf{D. Herzmann}, 2014: Streamflow Impacts of Biofuel Policy-Driven Landscape Change. PloS one, DOI: 10.1371/journal.pone.0109129
\item Anderson, C., R. Anex, R. Arritt, B. Gelder, S. Khanal, \textbf{D. Herzmann} and P. Gassman. 2013. Regional climate impacts of a biofuels policy projection. Geophysical Research Letters. 40(6), 1217-1222. http://dx.doi.org/10.1002/grl.50179
\item Mearns, L., S. Sain, L. Leung, M. Bukovsky, S. McGinnis, S. Biner, D. Caya, R. Arritt, W. Gutowski, E. Takle, M. Snyder, R. Jones, A. Nunes, S. Tuker, \textbf{D. Herzmann}, L. McDaniel, and L. Sloan. 2013. Climate change projections of the north american regional climate change assessment program (NARCCAP), 120(4), Climatic Change, 965-975
\item Khanal, S., R. Anex, C. Anderson, \textbf{D. Herzmann}, and M. Jha. 2013. Implications of biofuel policy-driven land cover change for rainfall erosivity and soil erosion in the United States. Global Change Biology BioEnergy., DOI: 10.1111/gcbb.12050
\item Breakah, T.M., R. Williams, \textbf{D. Herzmann}, and E. Takle. 2010. The Effects of Utilizing Accurate Climatic Conditions for Mechanistic-Empirical Pavement Design, a Case Study. Journal of Transportation Engineering, 137(1), 84–90.
\item Coleman, T.A., K. Knupp, and \textbf{D. Herzmann}. 2010. An Undular Bore and Gravity Waves Illustrated by Dramatic Time-Lapse Photography. J. Atmos. Oceanic Technol., 27, 1355-1361.
\item Coleman, T.A., K. Knupp, and \textbf{D. Herzmann}. 2009. The Spectacular Undular Bore in Iowa on 2 October 2007, Monthly Weather Review, 137(1), 495-503.
\item \textbf{Herzmann, D.E.}, J. Wolt, and R. Arritt. 2008. Representativity of a mesoscale network for weather-related factors governing pollen dispersal, International Journal of Biometeorology, 52(7), 617-624.
\item Williams, C.L., M. Liebman, J. Edwards, D. James, J. Singer, R. Arritt, and \textbf{D. Herzmann}. 2006. Patterns of Regional Yield Stability in Association with Regional Environmental Characteristics, Crop Science, 48(4), 1545-1559.
\item Cruse, R.M., D. Flanagan, J. Frankenberger, B.K. Gelder, \textbf{D. Herzmann}, D. James, W.F. Krajewski, M.M. Kraszewski, J.M. Laflen, J Opsomer and D. Todey. 2006. Daily estimates of rainfall, water runoff, and soil erosion in Iowa, Journal of Soil and Water Conservation, 61(4), 191-199.
\end{itemize}

\normalsize \sf
\normalsize \bf Presentations (Conference/Outreach)
\normalsize \sf
\begin{itemize}
%2014
\item Herzmann, D., 2015. Using the Mesonet. ISU Crops Team In Service, Ames, IA. Mar. 30, 2015.
\item Herzmann, D., 2014. IEM Dog and Pony Show. National Weather Service, Chicago, IL. Nov. 10, 2014.
\item Abendroth, L., D. Herzmann and L. Bunderson, 2014. Data management for regional transdisciplinary agricultural research: Approach and implementation. ASA-CSSA-SSSA Annual Meeting, Long Beach, CA. Nov. 2-5, 2014. https://scisoc.confex.com/scisoc/2014am/webprogram/Paper89677.html
\item Gassman, P.W., Y. Panagopoulos, C.L. Kling, M. Jha, J. Arnold, M. White, R. Srinivasan, S. Rabotyagov, M. Moskal, J. Anderson, R.E. Turner, N. Rabalais, R. Arritt and D. Herzmann, 2013. Development and application of SWAT modeling systems for two large U.S. corn belt river basins. Northwest A\&F University, Yangling, China. Oct. 29, 2014. 
\item Gassman, P.W., Y. Panagopoulos, C.L. Kling, M. Jha, J. Arnold, M. White, R. Srinivasan, S. Rabotyagov, M. Moskal, J. Anderson, R.E. Turner, N. Rabalais, R. Arritt and D. Herzmann, 2013. Development and application of SWAT modeling systems for two large U.S. corn belt river basins. Xi'an University of Technology, Xi'an, China. Oct. 28, 2014. 
\item Gassman, P., Y. Panagopoulos, R. Arritt, D. Herzmann, T. Campbell, C. Kling, M. Jha, R. Srinivasan, M. White and J. Arnold, 2014. Hydrologic, water quality and crop productivity impacts of climate change in the Ohio-Tennessee river basin. International SWAT Conference, Pernambuco, Brazil. July 28 - Aug. 1, 2014. http://swat.tamu.edu/conferences/2014/material/july-31/d2/
\item Herzmann, D., 2014. IEM Dog and Pony Show. National Weather Service, Minneapolis, MN. Mar. 4, 2014.
%2013
\item Gassman, P.W., Y. Panagopoulos, C.L. Kling, M. Jha, J. Arnold, M. White, R. Srinivasan, S. Rabotyagov, M. Moskal, J. Anderson, R.E. Turner, N. Rabalais, R. Arritt and D. Herzmann, 2013. Development and application of SWAT modeling systems for two large U.S. corn belt river basins. China Agricultural University, Beijing, China. Oct. 23, 2013. 
\item Herzmann, D., 2013. How much did it rain in Ames on June 25th, 2010?. ISU Geological and Atmospheric Sciences Fall 2013 Seminar Series. Sept. 3, 2013. Iowa State University. 
\item Panagopoulos, Y., P.W. Gassman, R. Arritt, D.E. Herzmann, T. Campbell, M.K. Jha, C.L. Kling, R. Srinivasan, M. White, and J.G. Arnold, 2013. Climate change effects on water pollution and crop production in the upper Mississippi river basin. CSCAP Annual Conference; Poster Symposium. West Lafayette, IN. July 29-Aug. 1, 2013. http://sustainablecorn.org/Publications/Posters.html
\item Herzmann, D.E., L.J. Abendroth, and L.D. Bunderson, 2013. CSCAP data management: Utilizing multiple tools to accomplish team science. CSCAP Annual Conference; Poster Symposium. West Lafayette, IN. July 29-Aug. 1, 2013. http://sustainablecorn.org/Publications/Posters.html
\item Herzmann, D. and L. Abendroth, 2013. CSCAP data management. REACCH Annual Conference. Portland, OR. 
\item Kling, C.L., P.W. Gassman, Y. Panagopoulos, R. Srinivasan, M.J. White, M.K. Jha, J.G. Arnold, R.E. Turner, S. Rabotyagov, J. Richardson, M.L. Moskal, N. Rabalais, R. Arritt, and D. Herzmann, 2013. An integrated modeling system to estimate Corn Belt Region nutrient load impacts on the Seasonal Gulf of Mexico Hypoxic Zone. ASABE Annual International Meeting, July 21-24, Kansas City, MO. 
\item Herzmann, D. and L. Abendroth, 2013. CSCAP data management: Utilizing multiple tools to accomplish team science. PINEMAP Annual Conference; Poster Symposium. Atlanta, GA. 
\item Herzmann, D., 2013. On the 2012 Drought and other items of interest. Professional Soil Classifiers of Iowa, Ames, IA. Mar. 2, 2013.
\item Herzmann, D. and L. Abendroth, 2013. CSCAP Data Management: Utilizing multiple tools to accomplish team science. REACCH Annual Conference. Poster Symposium. Portland, OR.
%2012
\item Herzmann, D. and L. Abendroth, 2012. Project collaboration and data management (whole team version). CSCAP Annual Conference. Wooster, OH. Aug. 7-9, 2012. 
\item Herzmann, D. and L. Abendroth, 2012. Project collaboration and data management (advisory board version). CSCAP Annual Conference. Wooster, OH. Aug. 7-9, 2012.
\item Herzmann, D., 2012. Musings on Storm Based Warnings. National Weather Service, Omaha, NE. Apr. 12, 2012.
%2011
\item Herzmann, D., 2011. Data management. CSCAP Annual Conference. Chicago, IL. Nov. 8-10, 2011. 
\item Wilke, A., J. Benning, C. Ingels, D. Herzmann, and L.W. Morton, 2011. Making climate change visible to farmers. CSCAP Annual Conference. Chicago, IL. Nov. 8-10, 2011.\\
http://www.sustainablecorn.org/doc/posters/2011/2011\_26-Wilke.pdf
\item Gelder, B., C. Anderson, R. Arritt, R. Anex, D. Herzmann, and P. Gassman, 2011. Environmental Impacts of Continental Scale Biofuel Production.  ASA-CSSA-SSSA Annual Meeting, Oct. 19, 2011. https://dl.sciencesocieties.org/publications/meetings/2011am/8440/66332
\item Herzmann, D., 2011. Musings on Storm Based Warnings. Central Iowa National Weather Association Annual Meeting, Des Moines, IA. Apr. 1, 2011.
%2010
\item Gelder, B., R. Anex, D. Herzmann, P. Gassman, C. Anderson, M. Jha, and R. Arritt, 2010. Climatalogical Impacts of Biofuel Production. ASA-CSSA-SSSA Annual Meeting, Nov. 3, 2010. https://dl.sciencesocieties.org/publications/meetings/2010am/7692/61374
\item Herzmann, D., 2009. IEM Potpourri. ISU Meteorology Seminar, Ames, IA. Sep. 22, 2009.
\item Herzmann, D., 2008. IEM: Using Open Source GIS Tools and Web Services to Disseminate Environmental Data. UNI GEOTree Seminar, Cedar Falls, IA. Feb. 28, 2008.
\item Herzmann, D., 2007. Iowa Enivornmental Mesonet. ISU Experiment Station Meeting, Ames, IA. Dec. 19, 2007.
\item Herzmann, D., 2007. IEM (Where's the Science?). ISU Meteorology Seminar, Ames, IA. Nov. 27, 2007.
\item Herzmann, D., 2007. IEM: Using Open Source GIS Tools and Web Services to Disseminate Environmental Data. University of Wisconsin GIS Day, Madison, WI. Nov. 7, 2007.
\item Herzmann, D., 2007. IEM: Supporting the real-time and research needs of the national severe weather community. Central Iowa National Weather Association Annual Meeting, Des Moines, IA. Mar. 24, 2007.
\item Herzmann, D., 2007. IEM Chat and other things that should interest the audience. National Weather Service Central Region SOO/WCM Conference, Kansas City, MO. Feb. 14, 2007.
\item Herzmann, D., 2006. NWS VTEC/Polygon Stuff. NWS Storm Based Warnings Workshop, College Station, TX. Dec. 5, 2006.
\item Herzmann, D., J. Wolt, R. Arritt, M. Westgate, and S. Goggi, 2006. Modeling Out Crossing Probabilities for Maize in Iowa. Agronomy Society Annual Meeting, Indianapolis, IN. Nov. 14, 2006.
\item Herzmann, D., 2006. IEM Update: From pollen to rivers to pavements to soils and a whole lot more. ISU Meteorology Seminar, Ames, IA. Oct. 10, 2006.
\item Herzmann, D., 2006. IEM Chat: A New Platform for Building NWS and Media Partnerships. Central Iowa National Weather Association Annual Meeting, Des Moines, IA. Mar. 25, 2006.
\item Herzmann, D. and R. Arritt, 2005. Using Open GIS web services to serve environmental data. Iowa-NASA Data Integration Workshop, Ames, IA. Apr. 20, 2005.
\item Herzmann, D. and R. Arritt, 2005. Using Open GIS web services to serve environmental data. American Meteorological Society Annual Meeting, San Diego, CA. Jan. 12, 2005.
\end{itemize}

% see cv_extension.tex
%
\normalsize \sf
\normalsize \bf Media
\normalsize \sf
\begin{itemize}
\item Herzmann, D. and W. Klein, 2013. Iowa environmental mesonet data used by thousands every day. Iowa State University Extension and Outreach. Sept. 30, 2014. http://www.extension.iastate.edu/article/iowa-environmental-mesonet-data-used-thousands-every-day
\item Love, O. (Daryl Herzmann cited), 2013. Wacky weather changing Iowans’ climate change perceptions. Cedar Rapids Gazette. Aug. 10, 2013. http://thegazette.com/2013/08/10/wacky-weather-changing-iowans-climate-change-perceptions/
\item Weather Brains (online netcast), 2012. Episode 334. http://weatherbrains.com/weatherbrains/?p=2869
\end{itemize}


\pagebreak
%____________________________________________________________________________
\end{document}
